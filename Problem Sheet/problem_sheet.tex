\documentclass[11pt,a4paper]{article} 
\usepackage{graphicx} 
\usepackage{amssymb, amsmath} 
\usepackage[margin=1cm]{geometry}
\usepackage[colorlinks=true,urlcolor=blue]{hyperref}

\begin{document}
\title{Introduction to Fortran: Problem Sheet}
\author{Chris Cooling}
\maketitle
\pagestyle{empty}
\thispagestyle{empty}

\section{Compilation}
\begin{itemize}
\item Obtain the code examples for this session
\begin{itemize}
\item Download from github: https://github.com/coolernato/Intro-to-Fortran.git
\end{itemize}
\item Find the “Compilation” directory
\item Source.f90 should be compiled as a single file
\item Source1.f90 and Source2.f90 should be compiled together
\item Compile and run the files by either:
\begin{itemize}
\item Compile it on your own computer and run it
\item Copy and paste it into https://www.onlinegdb.com , select “Fortran” in the top right and click “Run”
\end{itemize}
\end{itemize}

\section{My First Code}
\begin{itemize}
    \item Find the “My First Code” directory
    \item Compile and "run my\_first\_code.f90"
    \item Experiment with:
\begin{itemize}
    \item Changing the words in quotation marks following the print statement
    \item Adding more print statements
\end{itemize}
    \item You will need to recompile between making a change and running your program
\end{itemize}

\section{Variables}

\subsection{Mathematical Operators}
There are 5 identical cubes, each with a side length of 3.2m. Calculate and print:

\begin{itemize}
    \item The volume of one cube
    \item The area of all faces of one cube
    \item The volume of all cubes
    \item The area of all cubes
    \item The surface area to volume ratio of the cubes
\end{itemize}

100m3 of water is added to these cubes. One cube will be fully filled, before the enxt is filled and so on. Eventually there will be a number of completely filled cubes and a partially filled cube.

\begin{itemize}
    \item How many cubes are completely or partially filled?
    \item What volume of the partially filled cube is unfilled?
\end{itemize}


\subsection{Order of Operations}
\begin{itemize}
    \item Find the “Variables” directory
    \item Compile the “order\_of\_operations” file
    \item Write down what you expect the value of the different cases to be
    \item Run the file
    \item Check the results are what you expect
\end{itemize}



\subsection{Arrays}
A location in 3d Cartesian space may be represented by (x,y,z) coordinates. This may be represnted by a dimension 1 array with size 3.

\begin{itemize}
    \item Create a 1d array with three elements to represent Position A, which is at (1,2,1)
    \item Calculate the location of Position B, which has a displacement of (3,-4,1) from Position A
    \item Calculate the location of Position C, which is twice as far from the origin as Position B
    \item Calculate the location of position D, which is found by rotating position C 45$^{o}$ around the z axis. To rotate a location around the z axis, it may be multiplied by the matrix:
\end{itemize}

\begin{equation}
\begin{pmatrix}
\cos{(\theta)} & -\sin{(\theta)} & 0\\
\sin{(\theta)} & \cos{(\theta)} & 0 \\
0 & 0 & 1
\end{pmatrix}
\end{equation}

Extension:
\begin{itemize}
    \item Repeat the above, but with three points, all contained in a single two-dimensional array
    \item Initial points to use are (1,2,1), (-1,0,1) and (-3, -2, -2.5)
\end{itemize}




\end{document}