\documentclass[11pt,a4paper]{article} 
\usepackage{graphicx} 
\usepackage{amssymb, amsmath} 
\usepackage[margin=1cm]{geometry}
\usepackage[colorlinks=true,urlcolor=blue]{hyperref}

\begin{document}
\title{Introduction to Fortran: Problem Sheet}
\author{Chris Cooling}
\maketitle
\pagestyle{empty}
\thispagestyle{empty}

\section{Compilation}
\begin{itemize}
\item Obtain the code examples for this session
\begin{itemize}
\item Download from github: https://github.com/coolernato/Intro-to-Fortran.git
\end{itemize}
\item Find the “Compilation” directory
\item Source.f90 should be compiled as a single file
\item Source1.f90 and Source2.f90 should be compiled together
\item Compile and run the files by either:
\begin{itemize}
\item Compile it on your own computer and run it
\item Copy and paste it into https://www.onlinegdb.com , select “Fortran” in the top right and click “Run”
\end{itemize}
\end{itemize}

\section{My First Code}
\begin{itemize}
    \item Find the “My First Code” directory
    \item Compile and "run my\_first\_code.f90"
    \item Experiment with:
\begin{itemize}
    \item Changing the words in quotation marks following the print statement
    \item Adding more print statements
\end{itemize}
    \item You will need to recompile between making a change and running your program
\end{itemize}

\section{Variables}

\subsection{Mathematical Operators}
There are 5 identical cubes, each with a side length of 3.2m. Calculate and print:

\begin{itemize}
    \item The volume of one cube
    \item The area of all faces of one cube
    \item The volume of all cubes
    \item The area of all cubes
    \item The surface area to volume ratio of the cubes
\end{itemize}

100m3 of water is added to these cubes. One cube will be fully filled, before the enxt is filled and so on. Eventually there will be a number of completely filled cubes and a partially filled cube.

\begin{itemize}
    \item How many cubes are completely or partially filled?
    \item What volume of the partially filled cube is unfilled?
\end{itemize}


\subsection{Order of Operations}
\begin{itemize}
    \item Find the “Variables” directory
    \item Compile the “order\_of\_operations” file
    \item Write down what you expect the value of the different cases to be
    \item Run the file
    \item Check the results are what you expect
\end{itemize}



\subsection{Arrays}
A location in 3d Cartesian space may be represented by (x,y,z) coordinates. This may be represnted by a dimension 1 array with size 3.

\begin{itemize}
    \item Create a 1d array with three elements to represent Position A, which is at (1,2,1)
    \item Calculate the location of Position B, which has a displacement of (3,-4,1) from Position A
    \item Calculate the location of Position C, which is twice as far from the origin as Position B
    \item Calculate the location of position D, which is found by rotating position C 45$^{o}$ around the z axis (clockwise when viewed from above). To rotate a location around the z axis in this manner by an angle $\theta$, it may be multiplied by the matrix:
\end{itemize}

\begin{equation}
\begin{pmatrix}
\cos{(\theta)} & -\sin{(\theta)} & 0\\
\sin{(\theta)} & \cos{(\theta)} & 0 \\
0 & 0 & 1
\end{pmatrix}
\end{equation}

\begin{itemize}
    \item Calcualte the straight line distance between Positions A and D
\end{itemize}

Extension:
\begin{itemize}
    \item Repeat the above, but with three points, all contained in a single two-dimensional array
    \item Initial points to use are (1,2,1), (-1,0,1) and (-3, -2, -2.5)
\end{itemize}

\section{Conditionals}
\begin{itemize}
    \item In the Conditionals directory open the exercise\_conditionals.f90 file
    \item Write down what you think will be printed for each case
    \item Compile and run the file and check the results are what you expected
\end{itemize}

\section{Loops}
For each of the following tasks, write a piece of code using "do" or "do while" loops to achieve the following effects:
\begin{itemize}
    \item Print the numbers 1-20.
    \begin{itemize}
        \item If a number is a multiple of 3, print "Fizz" instead
        \item If a number is a multiple of 5, print "Buzz" instead
        \item If a number is a multiple of 3 and 5, print "FizzBuzz" instead
    \end{itemize}
    \item Calculate the value of the expression $\sum\limits_{n=1}^{\infty}0.5^{n} = 0.5+0.25+0.125+0.0625+\hdots$
    \item \begin{itemize}
        \item Use a variable to keep track of the increasing sum
        \item Each iteration of your loop should add a new value to the increasing sum
        \item When the difference between the increasing sum in one iteration of your loop and the previous is less than 0.0001, the loop should stop
        \item Print the final value
    \end{itemize}
\end{itemize}

\section{Subroutines}
Write a subroutine:


\section{Functions 1}
Write a function:
\begin{itemize}
    \item Named "cuboid_volume"
    \item Takes the real arguments "height", "width", "length"
    \item Returns the product of these values
\end {itemize}
Call this function with the following values and verify it produces the correct answers:
\begin{itemize}
    \item Height=2.0, width=3.0, length=1.5
    \item Height=3.0, width=10.0, length=4.0
\end {itemize}


\end{document}